\documentclass[11pt,letterpaper]{article}

% ---------------------------
% Packages (keep it sober)
% ---------------------------
\usepackage[margin=1in]{geometry}
\usepackage[T1]{fontenc}
\usepackage[utf8]{inputenc} % robust on Overleaf pdfLaTeX
\usepackage{lmodern}
\usepackage{microtype}
\usepackage[hidelinks]{hyperref}
\usepackage{booktabs}
\usepackage{array}
\usepackage{enumitem}
\usepackage{titlesec}

% Additions to ensure placeholders + long tracker table render reliably
\usepackage{xcolor}
\usepackage{mdframed}
\usepackage{longtable}

% ---------------------------
% Unicode safety (prevents "missing later text" due to smart quotes/dashes)
% ---------------------------
\DeclareUnicodeCharacter{201C}{``}
\DeclareUnicodeCharacter{201D}{''}
\DeclareUnicodeCharacter{2019}{'}
\DeclareUnicodeCharacter{2013}{--}
\DeclareUnicodeCharacter{2014}{---}
\DeclareUnicodeCharacter{2026}{\ldots}
\DeclareUnicodeCharacter{2022}{$\bullet$}
% Robust handling of NBSP (U+00A0) without relying on pasting an invisible char
\DeclareUnicodeCharacter{00A0}{~}

% ---------------------------
% Spacing / lists
% ---------------------------
\setlength{\parindent}{0pt}
\setlength{\parskip}{6pt}
\setlist[itemize]{leftmargin=1.4em, itemsep=3pt, topsep=3pt}
\setlist[enumerate]{leftmargin=1.6em, itemsep=3pt, topsep=3pt}

% ---------------------------
% Section styling (simple)
% ---------------------------
\titleformat{\section}{\large\bfseries}{\thesection.}{0.5em}{}
\titleformat{\subsection}{\normalsize\bfseries}{\thesubsection.}{0.5em}{}
\titleformat{\subsubsection}{\normalsize\bfseries}{}{}{ }
\titlespacing{\section}{0pt}{10pt}{6pt}
\titlespacing{\subsection}{0pt}{8pt}{4pt}

% ---------------------------
% Minimal environments
% ---------------------------
\newcommand{\commentheading}[1]{%
  \vspace{6pt}\noindent\textbf{Comment (#1)}\par
  \vspace{2pt}\hrule\vspace{6pt}
}
\newenvironment{commentblock}{\begin{quote}\small\itshape}{\end{quote}}
\newcommand{\responseheading}{%
  \vspace{2pt}\noindent\textbf{Response}\par
}
\newcommand{\locationheading}{%
  \vspace{2pt}\noindent\textbf{Location and evidence}\par
}
\newcommand{\LocationLine}{%
  \begin{actionbox}
  \textit{Location:} Section \_\_\_; page \_\_\_; lines \_\_\_.\par
  \textit{Evidence:} Figure \_\_\_ / Table \_\_\_ / Appendix \_\_\_.\par
  \end{actionbox}
}
% One environment to prevent broken/mismatched blocks for each comment.
\newenvironment{commentresponse}[2][7]{%
  \def\CommentResponseLines{#1}%
  \commentheading{#2}
  \begin{commentblock}
}{%
  \end{commentblock}
  \responseheading
  \ResponsePlaceholder[\CommentResponseLines]
  \locationheading
  \LocationLine
}

% ---------------------------
% Visible placeholders (boxes)
% ---------------------------
\mdfdefinestyle{plainbox}{
  linewidth=0.4pt,
  linecolor=black!35,
  backgroundcolor=white,
  roundcorner=0pt,
  innertopmargin=6pt,
  innerbottommargin=6pt,
  innerleftmargin=6pt,
  innerrightmargin=6pt,
  skipabove=6pt,
  skipbelow=6pt
}
\newenvironment{responsebox}{\begin{mdframed}[style=plainbox]}{\end{mdframed}}
\newenvironment{actionbox}{\begin{mdframed}[style=plainbox]}{\end{mdframed}}
\newcommand{\ResponsePlaceholder}[1][7]{%
  \begin{responsebox}
  \textit{[Insert response here.}\par\vspace*{#1\baselineskip}
  \end{responsebox}
}

% ---------------------------
% ID macros
% ---------------------------
\newcommand{\HE}[1]{HE-#1}
\newcommand{\RoneO}{R1-O}
\newcommand{\RoneM}[1]{R1-M#1}
\newcommand{\Ronem}[1]{R1-m#1}

% ---------------------------
% Title
% ---------------------------
\title{\textbf{Point-by-Point Response to Reviewers}\\
Manuscript ID: 5228655 (\emph{Environmetrics})}
\author{Antonio Aguirre \and Raquel Prado \and Bruno Sans\'o}
\date{Prepared January 2026}

\begin{document}
\maketitle
\tableofcontents

\section{Manuscript metadata}
\begin{tabular}{@{}>{\bfseries}p{0.23\linewidth}p{0.73\linewidth}@{}}
Manuscript title: & Bayesian Quantile-Based Correction and Synthesis of River Flow Forecasts \\
Journal: & \emph{Environmetrics} \\
Manuscript ID: & 5228655 \\
Decision letter date: & Sat, Jan 17, 2026 \\
Corresp. author: & Antonio Aguirre (\texttt{jaguir26@ucsc.edu}) \\
Co-authors: & Raquel Prado; Bruno Sans\'o \\
\end{tabular}

\section{How to read this response}
Each reviewer comment is quoted verbatim and followed by a response. For each response, we indicate the location(s) of the corresponding changes in the manuscript and summarize the supporting evidence (figures, tables, or appendices) when applicable.

\section{Decision letter (verbatim)}
% NOTE: This is the full letter text as pasted, including non-technical lines,
% so nothing is inadvertently omitted.

\begin{commentblock}
Dear Mr. Antonio Aguirre,

Thank you for submitting your manuscript ``Bayesian Quantile-Based Correction and Synthesis of River Flow Forecasts'' to our journal. After careful assessment, we have decided not to publish your manuscript in Environmetrics.

In fact, although the review team see merits in your manuscript, both of them require changes that exceed a major revision, but are open to considering a new submission

Hence, we have decided not to consider this manuscript further for publication. At the same time, I invite you to submit a new version of this paper.

If you decide to resubmit, please include the title and ID (5228655) of the current submission in the cover letter and address all the issues raised in a point-to-point rebuttal.

Comments from reviewers have been included at the bottom of this email to provide further details about this decision.

If you are considering submitting this manuscript to another journal, Wiley Editing Services may be able to help. Wiley Editing Services are available to all authors and offer expert help with manuscript preparation, including English language editing, translation, manuscript formatting, figure illustration, and graphical abstract design. To learn more, visit the Wiley Editing Services website.

You can also find free resources for writing and preparing your manuscript through Wiley Author Services.

We appreciate you considering Environmetrics for the publication of your research.
Kind regards,
Alessandro Fass\`o
Environmetrics
\end{commentblock}

\newpage

\section{Handling editor comments (point-by-point)}

\subsection{Summary of changes (handling editor)}
\begin{itemize}
\item[] To be completed: brief summary of major revisions addressing the handling editor's concerns.
\end{itemize}

\subsection{Handling editor text (verbatim preface)}
\begin{commentblock}
Handling editor comment:

This paper develops a Bayesian dynamical model approach for river flow forecasting. A reviewer and I have read the manuscript, and although this is an important problem, and the proposed methods are interesting, there are some serious shortcomings of the submitted manuscript that need to be addressed.
\end{commentblock}

\subsection{Specific handling editor comments}

\begin{commentresponse}[7]{\HE{1}}
The abstract purports a fast approach to the proposed DQLM framework, but that doesn't seem to be verified anywhere in the manuscript -- what is the computational time and complexity of the proposed method? Once it is fitted, does it need to be reestimated in future months or years? Is it feasible to do so?
\end{commentresponse}

\begin{commentresponse}[7]{\HE{2}}
One of the main issues with the paper is that the proposed method is not validated properly -- there are no competing alternatives (despite models, A, B and C being listed), the extended asymmetric Laplace is not really given validation either (how does this compare against just an asymmetric Laplace in this problem setting?). I imagine this hierarchical model is likely so complicated that no working hydrologists would understand, nor use it, without very clear evidence that it was substantially superior to other forecast products. The method should be validated against competing statistical models, as well as physical forecast ensembles to illustrate that the predictive distributions are sharper and better calibrated than simpler alternatives.
\end{commentresponse}

\begin{commentresponse}[7]{\HE{3}}
This brings me to another point, in that the hierarchical model is fairly detailed. A necessary revision would be to implement versions of the model without various hierarchical pieces -- e.g., Model C, what if the AR trend piece were removed? Or, Model B, what if retrospective and/or all states autoregressions were removed? In the current manuscript, the results look decent, but it's not really clear which part of the model is giving the most "bang for your buck", which is critical for end users to understand.
\end{commentresponse}

\begin{commentresponse}[7]{\HE{4}}
Again, on validation: include (along with comparisons against alternative models) proper scores quantifying the quality of the predictive distributions. CRPS is used in model estimation, apparently, but why are CRPS scores not reported out for the forecasting exercise? And, how do they change depending on regime? Because this is a quantile-based method, it is fair to include quantile scores as well (see Gneiting and Raftery 2007 for details).
\end{commentresponse}

\begin{commentresponse}[7]{\HE{5}}
It would be helpful for the reader to have access to the estimation and forecasting code for use in other basin applications.
\end{commentresponse}

\begin{commentresponse}[7]{\HE{6}}
Something that wasn't clear to me in the application was: is this in-sample, or out-of-sample validation? It appears that the model was fit to the data that is being used in validation, which isn't realistic. Please perform an appropriate cross-validation study to investigate the approach's actual ability to forecast unseen data.
\end{commentresponse}

\begin{commentresponse}[7]{\HE{7}}
The weighted combination of prior forecasts at differing horizons is curious to me. Presumably the most recent forecasts should be better with more available atmospheric data, whereas longer lead-time forecasts likely exhibit more bias. How does the model perform if you only use the most recent forecast (which I would expect to be the highest quality)?
\end{commentresponse}

\subsection{Handling editor transition line (verbatim)}
\begin{commentblock}
The following reviewer comments were taken into consideration during the peer review process.
\end{commentblock}

\section{Reviewer comments}

\subsection{Reviewer 1 (verbatim header + overview)}
\begin{commentresponse}[7]{\RoneO}
Reviewer comments:

Reviewer 1 Comments to the Author

This paper develops a Bayesian quantile regression approach, based on asymmetric Laplace distributions, to predict and forecast streamflow in the Santa Cruz region of the US. The work appears strong in the development of the Bayesian models and the novelty in the extension of inference methods is accepted. However, the improvement of forecasts through dynamical error correction is a common strategy and the novelty of this part may need to be elucidated further. The paper is framed as a forecasting paper, however the forecasting component is limited and robust forecast evaluation is absent. I recommend that some substantial revisions, mainly structural, are required before proceeding with publication. My comments below don't cover everything, but are intended as a push into reshaping:
\end{commentresponse}

\subsection{Summary of changes (Reviewer 1)}
\begin{itemize}
\item[]To be completed: brief summary of major revisions for Reviewer 1.
\end{itemize}

\subsection{Reviewer 1 --- Major comments (point-by-point)}

\begin{commentresponse}[7]{\RoneM{1}}
In the introduction and elsewhere, the "cascade of uncertainty" is not distinguished between meteorological uncertainty and hydrological uncertainty. This is not necessarily a major problem with a Bayesian model that combines sources of uncertainty, but I find the introduction and motivation less compelling when hydrological and meteorological concepts are mixed. For example, the paragraph that starts off "Hydrological predictions are often produced using physical models..." diverts to discussions of under-dispersion in ensembles and perturbation methods in weather forecasts. I would suggest restructuring the introduction to cover different sources of uncertainty before explaining how uncertainty can be consolidated in a Bayesian framework.
\end{commentresponse}

\begin{commentresponse}[7]{\RoneM{2}}
The links between the model formulation and the results are not clear to me. For example, three models (A, B and C) are developed in section 2. It would be helpful to explicitly link the models to the results being discussed in section 3.
\end{commentresponse}

\begin{commentresponse}[7]{\RoneM{3}}
The paper is maths heavy, so it may be beneficial to reduce the equations and mathematical details. The amount of detail given may be highly suitable for a dissertation, but some details are not necessary for a paper. For example, in section 2.7, PITs are described in detail, however, it is followed that CRPS is chosen for model selection instead and PITs are not used. Suggest removing PITs in this case. As another example, no quantile crossing is observed, however a two-step method is presented to "resolve" quantile crossing. I suggest reconsidering how much detail is needed for the Posterior Predictive Synthesis part.
\end{commentresponse}

\begin{commentresponse}[7]{\RoneM{4}}
If I have not misunderstood, only one short forecast has been evaluated for a moderate flood from 2022-12-25 to 2023-01-22. Is it possible to provide more evidence for the use of the model for forecasting high, low and moderate events? Or the continuous period 2023 to present? Going to the previous comment, PITs and CRPS have been used for model selection (section 2.7) rather than forecast verification. I would suggest that in an expanded evaluation of forecast performance that CRPS and PITs would be useful tools.
\end{commentresponse}

\begin{commentresponse}[7]{\RoneM{5}}
A robust discussion of how the method can be set up for fair forecast assessment using cross-validation might be required.
\end{commentresponse}

\subsection{Reviewer 1 --- Minor comments (point-by-point)}

\begin{commentresponse}[5]{\Ronem{1}}
Introduction: While it is true that "Hydrological predictions are often produced using physical models", it could be worthwhile noting that conceptual models are often more practical for prediction.
\end{commentresponse}

\begin{commentresponse}[5]{\Ronem{2}}
Section 2.1: I'm not familiar with the term "flexile quantile estimation". Would it be better to say "flexible quantile estimation"?
\end{commentresponse}

\begin{commentresponse}[5]{\Ronem{3}}
Section 2.3: Agree reanalyses have fewer problems that forecasts in terms of data gaps and uncertainties; however, it is important to note that fields like rainfall in ERA5 are filled out with short-run forecasts and are not completely deterministic.
\end{commentresponse}

\begin{commentresponse}[5]{\Ronem{4}}
Section 2.3: Introduce the USGS measurements and difference with reanalysis products. It may be useful to have a separate data and methodology section.
\end{commentresponse}

\begin{commentresponse}[5]{\Ronem{5}}
How is non-zero precipitation handled, is it through data censoring?
\end{commentresponse}

\begin{commentresponse}[5]{\Ronem{6}}
Section 3: lagged forecasts are weighted into an ensemble forecast for the target time rather than using the latest available forecast. In my experience, this can have the effect of degrading skill. Did the authors find this was beneficial?
\end{commentresponse}

\begin{commentresponse}[5]{\Ronem{7}}
Section 3.2: The title here is somewhat vague "General results". Given Section 3 is supposed to be on forecasting and these results appear to be about in-sample prediction, I suggest reorganising the headings and separating out model checking from forecasting.
\end{commentresponse}

\begin{commentresponse}[5]{\Ronem{8}}
Table 1 and 2: Check captions, should it be posterior mean or median?
\end{commentresponse}

\begin{commentresponse}[5]{\Ronem{9}}
Figures 8 and 9: Suggest some explanation around the approx. constant variance in Figure 8 and variable predictive distributions in Figure 9.
\end{commentresponse}

\section{Reviewer 2 comments (placeholder)}
The pasted email ends at ``Page 1 of 2'' and does not include Reviewer 2 comments. Insert Reviewer 2 comments verbatim here once available (e.g., from the second page of the email or the attachment \texttt{Revision\_Project\_1.pdf}).

\subsection{Summary of changes (Reviewer 2)}
\begin{itemize}
\item[]To be completed once Reviewer 2 comments are available.
\end{itemize}

\section{Change log / tracker (to be maintained)}
% Use ltablex so the tracker can span pages (tabularx cannot).
\clearpage
\small
\begin{longtable}{@{}>{\raggedright\arraybackslash}p{0.14\linewidth}%
>{\raggedright\arraybackslash}p{0.22\linewidth}%
>{\raggedright\arraybackslash}p{0.40\linewidth}%
>{\raggedright\arraybackslash}p{0.16\linewidth}@{}}
\toprule
\textbf{ID} & \textbf{Location} & \textbf{Change summary} & \textbf{Status} \\
\midrule
\endfirsthead

\toprule
\textbf{ID} & \textbf{Location} & \textbf{Change summary} & \textbf{Status} \\
\midrule
\endhead

\midrule
\multicolumn{4}{r}{\small Continued on next page} \\
\endfoot

\bottomrule
\endlastfoot

HE-1 &  &  &  \\
HE-2 &  &  &  \\
HE-3 &  &  &  \\
HE-4 &  &  &  \\
HE-5 &  &  &  \\
HE-6 &  &  &  \\
HE-7 &  &  &  \\
R1-O &  &  &  \\
R1-M1 &  &  &  \\
R1-M2 &  &  &  \\
R1-M3 &  &  &  \\
R1-M4 &  &  &  \\
R1-M5 &  &  &  \\
R1-m1 &  &  &  \\
R1-m2 &  &  &  \\
R1-m3 &  &  &  \\
R1-m4 &  &  &  \\
R1-m5 &  &  &  \\
R1-m6 &  &  &  \\
R1-m7 &  &  &  \\
R1-m8 &  &  &  \\
R1-m9 &  &  &  \\
\end{longtable}
\normalsize

\end{document}
